% !TeX root =../main.tex

\chapter*{Abstract}   \label{cha:Abstract}

Hier ein Beispieltext:

Due to the ambition to decarbonize the energy and heat sector an increasing amount of renewable energy sources is integrated in the existing grid. This leads to a change of requirements for the grid as well as the grid operators as the power flows change and the power generation also changes from centralized to decentralized. Therefore especially on the low and mid voltage stage an improved grid operating strategy as well as an extensive grid monitoring and control is needed. Doing so the grid can be operated in an optimized way to reduce the safety reserves and increase the penetration by renewable energy. Therefore it is necessary to detect and control critical situations inside the grid.\\
During the last years there have been a great interest and support by the government and the industry in research working on the optimization of distribution, e.g. low voltage grids.\\
During this thesis a framework for optimal operation of a distribution grid is developed purely based on artificial intelligence tools, which have been developing a lot over the last few years. The first part of this thesis will be about the extensive grid monitoring described above and a state estimation algorithm for distribution grids will be developed based on machine learning. This algorithm should be able to estimate the state of grid structures that are unobservable to a certain point without need to create specific "pseudo-measurements". For this deep learning as well as stochastic based approaches will be tested and evaluated against each other as well as against established estimating methods.\\
In the second part of this thesis an MPC voltage control for a low voltage distribution grid will be developed based on machine learning techniques. As most machine learning algorithms are purely databased the integration of a hybrid/ensemble model as well as extensive testing of stability and feasibility is necessary to make sure that the control always works in the desired way.\\
The algorithms developed in this thesis will finally be tested in a laboratory environment using hardware/software-in-the-loop testing combined with real-time- simulation which shows if the system is able to work in a close to reality environment.   