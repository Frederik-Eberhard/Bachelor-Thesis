% !TeX root =../main.tex

\chapter*{Abstract}   \label{cha:Abstract}
Due to the high amount of consumer electronic devices being used nowadays, \ac{SMPS} play a critical role in the electrical infrastructure. As millions of them are constantly used, drawing power from the electrical grid, even small amounts of power losses have a big impact on the overall energy expended. Because of this, maximising the efficiency of newly produced power supplies is crucial. However, the efficiency of a given \ac{SMPS} is affected by many different factors, complicating this task. Simulation, therefore, is a crucial step in the design process of \acp{SMPS}. With it, the vast amount of possible components, designs and settings can be reduced to a handful of optimal combinations suited for a specific application purpose. Currently, however, the models used for simulations of inductors, which are the key component in \ac{SMPS}, are not able to represent their losses well enough.\\
This thesis presents a novel approach to simulate the inductor of a \ac{SMPS} in LTspice, taking into account different sources of losses. Focusing on three different approaches to recreate the behaviour of multiple inductors, the non-linearities of these inductors are explained and implemented in LTspice. A buck converter is used to present the workings of \acp{SMPS}. Making use of provided switching element models, the created inductor models are inserted into the simulation of the buck converter. The same buck converter is then also recreated physically, enabling the direct comparison of the simulated and measured inductor behaviour. Operated at different switching frequencies, the losses of the inductor and the switching elements are compared. The results of the simulation are then shown to help in the prediction of the optimal switching frequency range for each buck converter, maximising efficiency.
