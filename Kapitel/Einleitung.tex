% !TeX root =../main.tex
\chapter{Introduction} \label{sec:intro}
Nowadays the average person owns a wide range of electronic devices, like phones, tablets and laptops. All of these devices rely on power supplies to be able to connect to the electrical grid to charge their internal batteries. These power supplies need to be small, light and efficient to meet the demands of the consumer. Energy usage and portability often are the main factors influencing their purchase. Because of this \ac{SMPS} are a crucial part of today's electronic infrastructure. Able to be compact, light and very efficient, they enable the transformation of \ac{AC} to \ac{DC} and vice versa as well as the increase or decrease of the voltage supplied to a connected device. All \acp{SMPS} are made up of three main types of components: the inductor, the capacitors and the switching elements. For most \ac{SMPS} the inductor stores and releases energy periodically. The way this energy is taken in and then released to the device at the output is dictated by the switching elements, typically in the form of \acp{MOSFET} or \acp{GaNFET}. The capacitors act as buffers at the input and output of the \ac{SMPS}, ensuring constant input and output currents and voltages. Depending on the construction of the \ac{SMPS}, referred to as its topology, and the control sequence of the switching elements, different transformations of the input to the output can be realised. Step-up converters, for example, output voltages higher than the input voltage. Similarly, step-down converters decrease the voltage. This deduction in the voltage in turn causes an increase in the current output, as power is ideally transmitted without losses from input to output. The components in the \ac{SMPS} however not being ideal, this is not possible. Instead, the efficiency, being the factor between incoming power and outgoing power, is attempted to be maximised. Observing the sources of power loss gives insight into possible optimisations. Mainly caused by the inductor and switching elements [Source], minimising their losses is the main objective to increase the \ac{SMPS}'s efficiency. As demonstrated later on in chapter \ref{sec:losses_in_the_switching_elements}, the losses in the switching elements are primarily proportional to the switching frequency $f_s$. A lower switching frequency results in lower losses in the switching elements. For the inductor-related losses, the relationship however is more complex. Due to the effects of the frequency response, saturation and hysteresis, detailed in chapter \ref{sec:losses_in_the_inductor}, the loss behaviour of the inductor is harder to describe and can vary substantially for different kinds of inductors.\\
Because of this, there have been multiple attempts to create accurate models of the inductor in simulation, to find optimal operating points. One of these recreated an inductor model in MATLAB, taking into account the effects of saturation and including a current-dependent resistance \cite{ECM_of_a_Ferrite_Core}. The resulting model is then simulated in a buck converter, a type of \ac{SMPS} also used in this thesis, and compared to measured behaviour. To improve the usability of the created inductor model, this thesis adapts the findings of the saturation behaviour and implements them into LTspice \cite{LTspice}, easing the integration of the model into existing \ac{SMPS} simulated circuits. With the focus on the switching frequency, the model is expanded to recreate the frequency behaviour of the inductor with current dependant behaviour taken into account through saturation.\\
A further attempt also created an inductor model in LTspice, adding a frequency-dependent series resistance to the default inductance and implementing a complex core model based on inductor characterisation parameters, called Steinmetz parameters \cite{Ridley_Inductor_Model}. Depending on the knowledge of these parameters to model the core, the presented approach is not feasible for this thesis' application. As many inductors used in \ac{SMPS} applications are produced by manufacturers which do not provide Steinmetz parameters and the measurement of these parameters exceeds the capability of most companies, this thesis presents an approach without the use of these parameters. Additionally, the frequency-dependant series resistance is expanded upon, by creating an \ac{ECM} which also takes the capacitive behaviour of the inductor into account.\\
This \ac{ECM} is equal to one of the two \acp{ECM} presented in another attempt, where the frequency-dependant behaviour of the inductor is combined with a different custom core loss model \cite{Comprehensive_SPICE_Model}. This core loss model again depends on the Steinmetz parameters, hindering its application to a broad range of inductors. Furthermore, the first \ac{ECM} of the paper is improved here by making use of a more accurate measurement setup and parameter calculation, eliminating the use for the second \ac{ECM}, which tries to work around these limitations.
Based on the prevalent use of a buck converter in these approaches to validate the created model, this thesis follows suit. Instead of using \acp{MOSFET} as the switching elements, this thesis specialises in the use of \acp{GaNFET} in the buck converter model, as they are more efficient and thereby better suited for \ac{SMPS} applications.\\

In chapter \ref{sec:fundamentals} the workings of \acp{SMPS} are explained by the example of a synchronous buck converter. Its structure and operating principle are introduced and the functions of the inductor and switching elements are discussed. Following this, the occurring losses are presented. Focusing first on the \acp{GaNFET}, their \ac{ECM} is presented and its switching frequency dependant loss behaviour is analysed. Then the inductor's sources of losses are discussed, differentiating between winding losses and core losses.\\

Chapter \ref{sec:models} then introduces three methods to model the inductor and its losses in LTspice. Each model relies on first measuring the physical inductor to then extrapolate the necessary data for its reconstruction in LTspice. Starting with an \ac{ECM} the frequency response of the inductor is recreated. Then the saturation behaviour is investigated and implemented into LTspice. Closing the chapter is the hysteresis modelling and its validation.\\

The setup for measuring and simulating the inductors is presented in chapter \ref{sec:measurement_setup}. Detailing the physical setup first it then moves on to two different buck converter implementations in LTspice used for the validation of the presented inductor model.\\

The results of first the measurement and then the simulation are presented in chapter \ref{sec:validation} and compared for different \acp{GaNFET} and inductors. Based on the comparison, the presented inductor model and its use for \ac{SMPS} optimisation is evaluated.\\
