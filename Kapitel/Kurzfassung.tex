% !TeX root =../main.tex

\chapter*{Kurzfassung}   \label{cha:Kurzfassung}
Aufgrund der großen Anzahl von elektronischen Geräten, die heutzutage verwendet werden, spielen Schaltnetzteile eine entscheidende Rolle in der elektrischen Infrastruktur. Mit Millionen von ihnen, die ständig Strom aus dem Stromnetz beziehen, haben selbst kleine Leistungsverluste große Auswirkungen auf den Gesamtenergieverbrauch im Netz. Aus diesem Grund ist die Maximierung des Wirkungsgrads von neu produzierten Schaltnetzteilen von entscheidender Bedeutung. Der Wirkungsgrad eines Schaltnetzteils wird jedoch von vielen Faktoren beeinflusst, wodurch dessen Maximierung erschwert wird. Das Simulieren des Schaltnetzteils ist daher ein entscheidener Schritt in dessen Entwurfsprocess. Die riesige Menge an möglichen Komponenten, Strukturen und Ansteuerungsweisen können somit auf eine Handvoll optimaler Kombinationen für einen bestimmten Anwendungszweck reduziert werden. Hierbei spielt die Spule als Schlüsselkomponente des Schaltnetzteils eine entscheidene Rolle bei der Verlustberechnung. Die derzeitigen Modelle, die für die Simulation von Spulen verwendet werden, sind jedoch nicht in der Lage, deren Verluste präzise darzustellen.\\
In dieser Arbeit wird ein neuartiger Ansatz zur LTspice Simulation von Spulen in Schaltnetzteilen vorgestellt, bei dem verschiedene Verlustquellen innerhalb der Spule berücksichtigt werden. Anhand von drei verschiedenden Ansätzen werden die Nichtliniearitäten mehrer Spulen erläutert und im Simulationsprogramm nachgebildet. Ein Abwärtswandler wird verwendet um die Funktionsweise von Schaltnetzteilen zu demonstrieren. Die erstellten Induktormodelle werden unter Verwendung der bereitgestellten Schaltelementmodelle in die Simulation des Abwärtswandlers eingefügt. Derselbe Abwärtswandler wird dann auch physikalisch nachgebaut , womit der direkten Vergleich des simulierten und gemessenen Spulenverhaltens ermöglicht wird. Bei unterschiedlichen Schaltfrequenzen werden die Verluste der Induktivität und der Schaltelemente verglichen. Die Simulationsergebnisse helfen dann bei der Vorhersage des optimalen Schaltfrequenzbereichs für einen gegebene Abwärtswandler, in dem die Effizienz maximiert wird.