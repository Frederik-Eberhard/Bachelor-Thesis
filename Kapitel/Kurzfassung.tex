% !TeX root =../main.tex

\chapter*{Kurzfassung}   \label{cha:Kurzfassung}

Wie bereits in der Schule gelernt, besitzen unterschiedliche Arten von Texten unterschiedliche förmliche Kriterien. Eine besondere Art eines Textes ist die wissenschaftliche Abschlussarbeit im Rahmen eines Studiums. Die Anforderungen an eine solche Arbeit ist in den unterschiedlichen Fachgebieten, wie z.B. Mechanik, Elektrotechnik, oder Verfahrenstechnik, zum Teil unterschiedlich, da sich historisch andere Konventionen ergeben haben. Aus diesem Grund soll Ihnen diese Vorlage für Ihre Abschlussarbeit dienen und gleichzeitig ein Art Wegweiser für die förmlichen sowie inhaltlichen Anforderungen bieten, die das Institut für elektrische Energietechnik an eine Abschlussarbeit stellt.

Dabei orientiert sich diese Vorlage an dem geforderten Aufbau einer Abschlussarbeit. Zu jedem Unterpunkt werden kurze Beispiele bzw. eine kurze Beschreibung zu den inhaltlichen Erwartungen gegeben. Besonders hervorzuheben ist dabei das Kapitel \ref{sec:fundamentals}, in dem eine Vielzahl an Beispielen und Latex Implementierungen ausgeführt wird. Außerdem ist in Kapitel \ref{sec:ausblick} eine Zusammenfassung für den Aufbau und den Inhalt der Arbeit zu finden. 

Sämtliche Texte in dieser Vorlage müssen durch Ihren Text, Überschriften, Abkürzungen usw. ersetzt werden. Ebenso müssen die Einstellungen in der Datei \enquote{Einstellung.tex} angepasst werden. 

