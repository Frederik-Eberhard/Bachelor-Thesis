%|============================================================================|
%| Pakete und Einstellungen                                                   |
%|============================================================================|

% ---- Eigene Packages----
\usepackage[colorinlistoftodos]{todonotes}
% ------------------------

\usepackage{ifthen}

% ---- Spracheinstellungen ----
%\usepackage[ngerman]{babel}		% Neue dt. Rechtschreibung, Silbentrennung (auch im Literaturverzeichnis)
\ifthenelse{\equal{\Sprache}{0}}
	{\usepackage[ngerman]{babel}
	\usepackage[ngerman]{isodate}
     \usepackage{ziffer} % Stellt sicher, dass im Deutschen keine Spaces zwischen Kommas in Zahlen vorkommen (1, 155 wird 1,155}
	}{}
\ifthenelse{\equal{\Sprache}{1}}
	{\usepackage[ngerman,english]{babel}
	\usepackage[ngerman,english]{isodate}
	}{}


\usepackage[utf8]{inputenc}		% Deutsche Sonderzeichen/ Umlaute
\usepackage{helvet}             % Schriftart Helvetiva
% Schriftart Helvetica für Überschriften und Text
\renewcommand\familydefault{phv}
\usepackage{csquotes}
\usepackage[T1]{fontenc}        % Stellt sicher, dass im PDF Umlaute gefunden werden
\usepackage{textcomp}           % Zusätzliche Symbole
\usepackage{setspace}
\setlength{\emergencystretch}{2em}		% erlaubt größeren Leerraum zwischen Wörtern, damit bei Blocksatz besser umgebrochen werden kann (reduziert overfull\hbx-Meldungen)
%In diesem Package lassen sich Silbentrennungen für Zeilenumbrüche von unbekannten Wörtern definieren
\hyphenchar\font=\string"7F
\hyphenation{Wirk-leis-tungs-un-gleich-ge-wicht Mo-men-tan-re-ser-ve}					


% ---- Mathe, Physik etc. ----
\usepackage[
            fleqn                                     % Linksbündige Formeln
            ]{amsmath}                                % Zur Abbildung der mathematischen Symbole und Formeln
\usepackage[fixamsmath,disallowspaces]{mathtools}     % Erweitert amsmath und behebt einige Fehler
\usepackage{nccmath}
\usepackage{amsfonts}                                 % Mathematische Schriftarten
\usepackage{array} 									  % array-Umgebung (für Matrizen etc)
\usepackage{bm} 		% Fett schreiben im Mathemodus über \bm
%\sisetup{locale = DE}
\usepackage{siunitx}	% sorgt für eine korrekte Darstellung von Einheiten; Umwandlung in andere Systeme als SI möglich
%\usepackage{ziffer}   	% Verhindert den kleinen Platz nach Kommas im Mathemodus
\ifthenelse{\equal{\Sprache}{0}}{\sisetup{locale = DE}}{\sisetup{locale = EN}}


% ---- Kopfzeile ----
% Seitenstil, verhindert nur Großbuchstaben im Header, "plainheadsepline" erstellt auch eine Linie bei Chapter
\usepackage[automark, plainheadsepline, headsepline]{scrlayer-scrpage}


% ---- Abbildungen ----
\usepackage[
           singlelinecheck=false,	% Linksbündige Captions
           labelfont=bf
           ]
           {caption}[2008/08/24]	% Erzeugt eine numerierte Bemerkung innerhalb der figure- und table-Umgebungen
\usepackage{subcaption}				% Captions für Unterbilder
\usepackage{graphicx,psfrag}        % Einbindung von Abbildungen, EPS Benutzung
\usepackage{flafter} 				% Verhindert, dass eine Abbildung im Text vor der ersten Nennung auftaucht
																																		 

\usepackage{here}       % Zum erzwingen der Bildplatzierung mit [H] (am besten nicht erzwingen sondern fließen lassen und nur machen wenn der satzspiegel einer Seite gut ist!!)
\usepackage{epstopdf}   % Wandelt *.eps-Dateien on the fly in PDF´s um, damit die Konvertierung direkt in PDF klappt


% ---- Gleitparamter für optimales Gleiten von Abb. und Tab. ----
\setcounter{topnumber}               {1}
    \setcounter{bottomnumber}        {1}
    \renewcommand{\floatpagefraction}{0.8}
    \renewcommand{\topfraction}      {0.8}
    \renewcommand{\bottomfraction}   {0.5}
    \renewcommand{\textfraction}     {0.15}
    \makeatletter
      \setlength{\@fptop}{0pt}
    \makeatother

	
% ---- Fussnoten ----
\usepackage{chngcntr}             	% Erlaubt das Erstellen und Ändern eigener Counter
\counterwithout{footnote}{chapter}  % Lässt die Nummerierung der Fußnoten auch über Kapitel hinweg fortlaufen, wenn nach jedem neuen Kapitel bei 1 angefangen werden soll -> auskomemntieren
\usepackage{footmisc}               % Fußote eingerückt, in der nächsten Zeile wieder linksbündig0


% ---- ANHANG ----
\usepackage[toc,page,title,header]{appendix} 			%toc: Anhang wird im Inhaltsverzeichnis gezeigt
\addto\captionsngerman{\let\appendixtocname\appendixname
\let\appendixpagename\appendixname} 					%Das deutsche Wort Anhang wird statt Appendix angezeigt


% ---- Abkürzungen ----
\usepackage{acronym} % Abkürzungen werden in der Datei 'abkuerzungen.tex' definiert
% \acro{Kurzname}{Langname} in der Umgebung \begin{acronym} (s. Datei)
% Bsp.: \acro{WKA}{Windkraftanlage}
% Im Text schreibt man statt 'WKA' '\ac{WKA}'. An der Stelle des ersten Aurufs steht
% automatisch Windkraftanlage (WKA) und ab da an nur noch WKA
% Der Plural von Abkürzungen wird mit \acp aufgerufen. Wenn der Plural des Wortes nicht
% durch ein 's' erzeugt wird (englisches Package) muss er gesondert definiert werden
% \acrodefplural{WKA}[WKA]{Windkraftanlagen} - der Plural von WKA wird hier ebenfalls mit WKA
% abgekürzt.


% ---- Verwaiste Zeilen ----
\usepackage[all]{nowidow}	% Verhindert, dass eine einzelne Zeile (Waisenzeile) am Ende einer Seite (nach einem Absatz), bzw. am Anfang einer Seite (zum Ende eines Absatzes) auftaucht

% ---- TABELLEN ----
\usepackage{color}					% für Farben im allgemeinen
\usepackage{colortbl}				% für die Hintergrundfarbe einzelner Zellen in Tabellen
\usepackage{multirow}				% Erlaubt es, innerhalb einer Tabelle mehrzeilige/spaltige Zellen zu erstellen
\usepackage{rotating}				% Schrift in Tabellen vertikal ausrichten durch \begin{sideways} TEXT \end{sideways}
\usepackage{tabularx} 				% Tabellen mit variabel breiten Spalten (gekennzeichnet durch X), so dass die Tabelle immer auf die Breite der Seite gestreckt wird
\usepackage{xltabular}              % Tabellen brechen selbstständig und verlaufen über mehrere Seiten

% ---- LITERATURVERZEICHNIS ----
% Das Literaturverzeichnis wird über biblatex mit dem Backend biber erstellt.
% Die Literatur wird in der Datei literatur.bib abgelget, die bspw. mit JabRef bearbeitet werden kann
% Um die Datei richtig kompilieren zu können, muss die Art der Bibliographie (Biblographie->Art) auf biblatex gestellt sein
% Außerdem muss unter (Optionen->Texstudio konfigurieren->Erzeugen) das Standard Bibliographieprogramm auf 'Biber' gestellt sein.
% In anderen Latex-Umgebungen sollten sich ähnliche Einstellungen finden lassen

% \usepackage[backend=biber,  	% Das Backend zum kompilieren der .bib-Datei
% 			style=ieee, 		% Der Stil indem das Literaturverzeichnis angezeigt wird
% 			citestyle=numeric,
% 			urldate =comp] 	% Zitate im Text werden durchnumeriert und entsprechend im Verzeichnis geordnet
% 			{biblatex}
% \renewbibmacro*{author}{\printnames{author}} % Verhindert, dass gleiche Autorennamen hintereinander durch Striche ersetzt werden (ist im ieee-Style implementiert)

%\addbibresource{Literatur/literatur.bib} % Pfad zur .bib-Datei


% ---- Farbdefinitionen ----
\definecolor{dunkelgrau}{rgb}{0.8,0.8,0.8}
\definecolor{hellgrau}{rgb}{0.95,0.95,0.95}
\definecolor{hellblau}{rgb}{0.8,0.8,0.8}


% ---- DIVERSES ----
\usepackage{pdfpages}		% PDF-Seiten im Anhang einbinden
\raggedbottom				% SOLL BEI TWOSIDED VERHINDERN DASS ABSTÄNDE ZU GROß SIND
\usepackage{tikz}
\usetikzlibrary{positioning}
\usetikzlibrary{pgfplots.groupplots}
\usetikzlibrary{patterns} %for patterns in bar plot
\usetikzlibrary{spy}
\usepackage{amssymb}
% ======================== Optionale Pakete ===========================
% an dieser Stelle können eigene Pakete eingebunden werden.
% bereits vorhandene Pakete können bei Bedarf einkommentiert werden

%\usepackage[official]{eurosym}  % Offizielles Euro-Zeichen

%\usepackage{multirow}			 % Erlaubt es, innerhalb einer Tabelle mehrzeilige/spaltige Zellen zu erstellen

%\usepackage{colortbl}   		 % Tabellellinien in anderen Farben

%\usepackage{overpic}			 % Bilder mit anderen Bildern überlagern (Bspw. für Legenden)

%\usepackage{pdflscape}		     % einzelne Seite im Querformat

%\usepackage{listings}					% Führt Listings zur Darstellung von Code ein					
%\renewcommand{\lstlistingname}{Programmcode}	% Ändert Titel der Listings

\usepackage{pgfplots} % TICKZ einbinden (PGF)
\pgfplotsset{compat=1.18}
\usepackage{algorithm}	% Include Pseudocode, algorithmic
\usepackage{algpseudocode}

\ifthenelse{\equal{\Sprache}{0}}
	{% Abbildung
	\newcommand{\figref}[1]{Abb.~\ref{#1}}
	% Gleichungs--Referenzen:
	\renewcommand{\eqref}[1]{Gl.~(\ref{#1})}
	}{}
\ifthenelse{\equal{\Sprache}{1}}
{% Abbildung
	\newcommand{\figref}[1]{Fig.~\ref{#1}}
	% Gleichungs--Referenzen:
	\renewcommand{\eqref}[1]{Eq.~(\ref{#1})}
}{}
\newcommand{\tabref}[1]{Tab.~\ref{#1}}
% Algorithmus --Referenzen: 
\newcommand{\algoref}[1]{Alg.~\ref{#1}}
% ==================================================================


% ---- HYPERREF ---- !WICHTIG! - {hyperref} als letztes Paket laden
\usepackage{hyperref}	% Verlinkung von farbigen Referenzen im Text, z.B. Verlinkung direkt zu einer Fußnote oder Literatur
\hypersetup{
  colorlinks=true,	% aktiviert farbige Referenzen
  linkcolor=black,	% Farbe der Verlinkungen innerhalb des Textes, z.B. Kapitel, Fußnote etc.
  citecolor=black,	% Farbe der Literatur-Refs
  urlcolor=black,	% Farbe einer URL im Literaturverzeichnis
  breaklinks=true,	% urls im Literaturverzeichnis umbrechen
% pdfpagemode=None,	% PDF-Viewer startet ohne Inhaltsverzeichnis et.al.
% pdfstartview=FitH % PDF-Viewer benutzt beim Start Seitenbreite des Bildschirms
            }
\usepackage{geometry}                               % Erlaubt das ändern des Layouts
  \geometry{                                        % Layout des Dokuments (Muss angepasst werden!)
            a4paper,                                % Papierformat
            includeheadfoot,                        % Einfügen von Kopf und Fußzeile
            inner=3cm,                             	% Innerer Rand
            outer=2.5cm,                            % Äußerer Rand
            top=1.5cm,                              % Höhe Kopfzeile
            bottom=1.5cm,                           % Höhe Fußzeile
            }       

% Erstellen von SI-Einheiten, die in neueren Paketen nicht mehr verfügbar sind.
\DeclareSIUnit\bar{bar}
%\DeclareSIUnit\ohm{$Omega$}

