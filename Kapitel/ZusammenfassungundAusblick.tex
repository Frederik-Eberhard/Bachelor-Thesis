% !TeX root =../main.tex

\chapter{Zusammenfassung und Ausblick} \label{sec:ausblick}

Stellen Sie hier auf ein paar Seiten die wesentlichen Ergebnisse und Kernaussagen Ihrer Arbeit noch einmal zusammen. Sie können sehr konkret sein, da alles vorher gesagt ist. Versuchen Sie, Ihre Ergebnisse selbst zu würdigen. Seien Sie nicht zu bescheiden, machen Sie aber aus einer Mücke auch keinen Elefanten. Geben Sie Erfahrungen an, die Sie gemacht haben, die für den Leser oder nachfolgende Studierende in diesem Themenbereich möglicherweise von Interesse sein könnten. In der Zusammenfassung wird nichts Neues mehr aufgegriffen, ausgewertet oder beschrieben. Wie der Name schon sagt, werden lediglich bekannte Aspekte zusammengefasst.
Die Arbeit sollte am Ende im Allgemeinen münden. Diese Funktion kann ein Ausblick gut übernehmen. Hierhin gehören Anregungen für eine konkrete Weiterentwicklung der Arbeit, z.B. neue Anwendungsbereiche, funktionale Erweiterungen, Modifikationen aus erkannten Problemen heraus usw., aber auch generelle Anmerkungen zur Einschätzung der zukünftigen Entwicklung der in der Arbeit behandelten Thematik.

