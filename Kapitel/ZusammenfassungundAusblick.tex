% !TeX root =../main.tex
%--> SMPS, switching frequency, buck converter
%--> 3 approaches were tried, ecm, saturation, hysteresis. --> ecm and saturation worked, hysteresis didn't
%--> buck converter realised, chosen components from manufacturers
%--> realised model only able to roughly approximate ideal switching frequency. not as hoped. 
%--> Future Work: using other approaches, like behavioural sources to imitate the inductor behaviour
%-->              Using different solvers, which might yield better results
%-->              Comparison for different topologies
%-->              Proprietary solvers exist for certain inductors, modelling their behaviour in SMPS, but they aren't general / only describe the inductor
%Stellen Sie hier auf ein paar Seiten die wesentlichen Ergebnisse und Kernaussagen Ihrer Arbeit noch einmal zusammen. Sie können sehr konkret sein, da alles vorher gesagt ist. Versuchen Sie, Ihre Ergebnisse selbst zu würdigen. Seien Sie nicht zu bescheiden, machen Sie aber aus einer Mücke auch keinen Elefanten. Geben Sie Erfahrungen an, die Sie gemacht haben, die für den Leser oder nachfolgende Studierende in diesem Themenbereich möglicherweise von Interesse sein könnten. In der Zusammenfassung wird nichts Neues mehr aufgegriffen, ausgewertet oder beschrieben. Wie der Name schon sagt, werden lediglich bekannte Aspekte zusammengefasst.
%Die Arbeit sollte am Ende im Allgemeinen münden. Diese Funktion kann ein Ausblick gut übernehmen. Hierhin gehören Anregungen für eine konkrete Weiterentwicklung der Arbeit, z.B. neue Anwendungsbereiche, funktionale Erweiterungen, Modifikationen aus erkannten Problemen heraus usw., aber auch generelle Anmerkungen zur Einschätzung der zukünftigen Entwicklung der in der Arbeit behandelten Thematik.


\chapter{Summary and  Future Work} \label{sec:ausblick}
The goal of this thesis was to create an LTspice inductor model, which improves on existing models used for operations in \ac{SMPS}. This model can then be used to approximate the losses occurring in \ac{SMPS}, primarily buck converters, and help to find the ideal switching frequency to maximize efficiency. Three implementation approaches were considered based on the physical behaviour of inductors.\\
Firstly, to recreate the frequency response of the inductor an \ac{ECM} is created. Determining the frequency response was done using the "Bode 100", resulting in low noise measurements allowing for precise extraction of the inductance, capacitance and resistances of the \ac{ECM}. This model was then validated by observing its simulated frequency response, which closely matched the measured data. \\
Secondly, the saturation behaviour is included, which models the current dependent inductance of the inductor. For it a further measurement device was used, the "Power Choke Tester". Mathematical functions were matched to its output curve and imported into LTspices \textit{flux}-command. The simulated inductor was then subjected to sinusoidal currents of different amplitudes, showing how its inductance followed the desired saturation curve. To consolidate its accuracy the triangular ripple current at different \ac{DC} offsets was measured for different buck converters and compared to the simulated inductors subjected to the same load. Here the inductors however were not able to accurately recreate the ripple currents amplitude, only showing similarities in the global behaviour. While the inclusion of saturation does improve the inductor model in LTspice, it is not suited for use cases in \ac{SMPS}, inducing errors into the simulation. 
Lastly, the B-H curve of the inductor is reconstructed, to account for core losses and \ac{DC} bias. A custom inductor and measurement setup is built, to reduce the number of unknowns and capture the properties of the hysteresis curve. As the measurement is too noisy to extract precise values, 
the hysteresis curve provided by the manufacturer is used instead. With its parameters imported into LTspice, a hysteresis curve similar to the provided one can be measured. When measuring the simulated saturation behaviour of the inductor, however, it barely correlates with the measured saturation, thereby deeming this model to be unsuited for accurate \ac{SMPS} recreation.\\
After having reached the limits of the inbuilt saturation and hysteresis models of LTspice, the \ac{ECM} is tested against a physical buck converter. Making use of the provided \ac{GaNFET} models for LTspice simulation a recreation of the buck converter is implemented in LTspice. Its efficiency behaviour at different switching frequencies is compared with the physical realisation, showing a non-negligible difference between the two. The simulated efficiency approximates the behaviour of the true efficiency but is only reliably able to define a range of switching frequencies, which contains the ideal switching frequency. In conclusion, the options LTspice presents to model inductors are limited to certain use cases. While frequency response, saturation and hysteresis can all be recreated to a certain accuracy, they only represent the inductor's behaviour in certain cases and are not able to reflect accurately, how inductors in \ac{SMPS} behave.\\

The approaches used in this thesis do not exhaust the possible ways of modelling inductors with LTspice. A different approach might be to use behavioural voltage and current sources to model the effects of the \ac{DC} biased ripple current more accurately. The \ac{ECM} could also be expanded to include function-defined inductors, capacitors and resistors in an attempt to recreate the core losses. Furthermore, the different solvers of LTspice's competitors like PSpice might better reflect the inductor's behaviour in \acp{SMPS}.





