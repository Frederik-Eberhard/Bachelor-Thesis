% !TeX root =../main.tex

\chapter{Die Endphase der Arbeit}

Der oder die Anhänge bestehen aus voneinander unabhängigen Ergänzungen zu bestimmten Aspekten, die in der Arbeit nicht weiter vertieft werden können oder sollen, aber in einem gewissen Zusammenhang für den Leser relevant sind. Typische Beispiele sind Datenblätter von Betriebsmitteln, vollständige Schnittstellenbeschreibungen, umfangreiches, tabellenartiges Datenmaterial, Syntaxdiagramme oder Grammatiken, Konfigurationsparameter und -dateien, sowie ein Glossar. Es ist Pflicht, Quellcode, Fotos, Messergebnisse u. ä. (inkl. der Rohdaten) und damit in Zusammenhang stehende Dokumente auf Datenträgern in einer eingeklebten Tasche bereitzustellen.




\textbf{Vervollständigung} \\
Das letzte Korrekturlesen sollten Sie zwei Mal mit „unterschiedlicher Brille“
durchführen:
\begin{enumerate}
    \item Aus formaler Sicht: Hierbei sollten Sie bewusst die inhaltliche Sicht
ausblenden und sich nur auf Formales konzentrieren: Rechtschreibung,
Bildgrößen, etc. Die Erfahrung zeigt, dass man beim inhaltlichen Lesen gerne
interpoliert, formale Fehler werden sehr leicht überlesen.
    \item Aus inhaltlicher Sicht: Sind die Zusammenhänge logisch, fehlen Verweise, sind alle Formeln mathematisch richtig oder sind sonstige inhaltliche Fehler und Ungenauigkeiten zu finden.
\end{enumerate}

Bei der Bewertung der abschließend eingereichten Arbeit spielt der „erste Eindruck“, welcher meist durch formale Aspekte beeinflusst wird, immer eine Rolle. Erzeugen Sie eine ansprechende äußere Form, indem Sie abschließend die gesamte Arbeit noch einmal auf Details der Darstellung und Formatierung überprüfen und somit ein konsistentes professionelles Erscheinungsbild erzeugen.

Hier bietet es sich an, die Arbeit, wenn möglich, von einer oder mehreren weiteren Personen Korrektur lesen zu lassen. Eine „objektivere“ Meinung kann abschließend dabei helfen, letzte Unstimmigkeiten auszumerzen.



\chapter{Typische Fehler in Abschlussarbeiten}
Abschließend werden noch einige typische Fehler beschrieben, wie sie in Abschlussarbeiten vorkommen. Es soll eine Sensibilisierung für diese Fehler stattfinden und gleichzeitig Hilfestellungen zur Vermeidung gegeben werden.
\begin{itemize}
    \item  \textbf{Zu ausführliche Grundlagen} \\
Die Grundlagen können leicht Lehrbuchcharakter annehmen. 
Wie kommt es dazu? Studierende neigen dazu, gerade im Grundlagenkapitel viel allseits Bekanntes zu schildern – vielleicht aus der Furcht heraus, dass der Text sonst zu kurz würde. Dies sollten Sie vermeiden. Die Abschlussarbeit soll kein Lehrbuch sein. Sie können davon ausgehen, dass der Leser die gängigen Kenntnisse Ihrer Disziplin hat. Einem Energietechniker brauchen Sie z.B. nicht zu erläutern, wie Strom und Spannung definiert sind. Sie sollten nur die spezifischen Dinge schildern, welche nicht im üblichen Kanon enthalten sind. Es ist bekannt, dass dieser Punkt oftmals Schwierigkeiten verursacht, daher wird in solchen Fällen die Absprache mit dem Betreuer empfohlen. Tatsächlich gibt es hierzu auch unterschiedliche Standpunkte, wie puristisch eine Arbeit geschrieben werden sollte. 

\item	\textbf{Füllwörter} \\
Es ist in Leseproben von Abschlussarbeiten oftmals zu lesen: „Es gibt ziemlich viele ...“, „Der eigentliche Kern ist ...“, „Die Lücke ist sehr groß ...“. Allgemein ist anzuraten, solche Füllwörter („ziemlich, eigentliche, sehr“) zu streichen. Der Text wirkt dadurch prägnanter, überzeugender, und der Inhalt ist derselbe. Machen Sie den Test: Wenn Sie gerne Füllwörter verwenden, streichen Sie diese aus einem Absatz und machen Sie den „Vorher-Nachher-Vergleich“.

\item	\textbf{Monotoner Text ohne Visualisierung}\\
Verwenden Sie Abbildungen zur Visualisierung. Denn: 
\item[]\begin{itemize}
\item Das Auge des Lesers braucht ab und zu Abwechslung vom Text. 
\item Komplexe Sachverhalte lassen sich meist mit Abbildungen, z.B. Ablaufdiagrammen, einfacher erklären. Übertreiben Sie es aber auch nicht. Selbst in „wissenschaftlicher“ Literatur werden in Abbildungen teilweise Trivialitäten geschildert, was vermieden werden sollte.
\end{itemize}

\item \textbf{Ergebnisse nicht verständlich} \\
Dies ist ein besonders kritischer Fall, welcher aus verschiedenen Gründen entstehen kann:
\item[]\begin{itemize}
\item Falsche Reihenfolge: Sie erklären etwas, das etwas voraussetzt, was Sie erst zu einem späteren Zeitpunkt erklären. Wie kommt es? Während des Schreibens haben Sie bereits „alles“ im Kopf, Sie sind sich nicht bewusst, dass Sie den anderen Punkt erst später erwähnen.
\item In Ihrer Argumentationskette gibt es Lücken. Sie selbst interpolieren diese Lücke, haben Hintergrundwissen im Kopf, für den Leser trifft dies jedoch nicht zu.
\item Ein Satz ist mehrdeutig, verschiedenartig interpretierbar – ein sehr häufiger Grund. Sie greifen sich Ihre Interpretation heraus, der Leser seine; oder er rätselt, welche hier passt.
\end{itemize}

\item	\textbf{Termini nicht identisch verwendet}\\
Verwenden Sie für dasselbe immer denselben Begriff. Beispiel:
Einmal verwenden Sie „Workflow“, einmal „Prozess“. Wenn es sich immer um Workflows handelt, verwenden Sie immer dieses Wort. Ansonsten rätselt der Leser, ob bei „Prozess“ wohl etwas Allgemeineres oder Anderes gemeint ist. Die Tendenz, Worte abzuwechseln wurde im Deutschunterricht vermittelt, wodurch eine lebhaftere, abwechslungsreichere Sprache propagiert wird. Bei technischen Texten ist dies jedoch verwirrend.

\item	\textbf{Bilder zu klein} \\
Ihr Text enthält Bilder (Grafiken, Abbildungen), meist ist darin ebenfalls Text in Form von Beschriftungen enthalten. Häufig sind Bilder unnötig klein und die Texte damit kaum lesbar. Dies strengt den Leser unnötigerweise an. Achten Sie also auf den visuellen Eindruck der Bilder und auf die leichte Lesbarkeit der Texte. Eventuell sind Textteile unnötig und können entfernt werden. Korrekte Achsenbeschriftungen sind hingegen unabdingbar.

\item	\textbf{Quellennachweise von Bildern} \\
Wenn Sie ein Bild (Abbildung, Grafik) von anderer Seite (Buch, Zeitschrift, Internet, firmeninterner Bericht oder Präsentation) übernehmen, müssen Sie die Quelle kenntlich machen. Sonst wäre davon auszugehen, dass Sie das Bild selbst erstellt haben. Hierbei wird der Quellenverweis direkt in der Bildunterschrift genannt. 

\item	\textbf{Fehlender Verweis auf Abbildungen im Text} \\
Auf alle in der Arbeit abgedruckten Abbildungen muss im Text verwiesen werden. Ist dies nicht der Fall, so ist der Zusammenhang, in dem die Abbildung steht, nicht klar ersichtlich. Es ist dabei zu beachten, dass die Verweise innerhalb der Arbeit vor der betreffenden Abbildung zu platzieren sind. 

\item	\textbf{Verwendung von Zahlen im Text} \\
Die Zahlen von 1-12 werden im Text ausgeschrieben (z.B. „Der Windpark besteht aus drei Windkraftanlagen.“). Höhere Zahlen werden nicht ausgeschrieben (z.B. „Der Windpark besteht aus 13 Windkraftanlagen.“). Ausnahmen können gemacht werden, wenn eine Einheit auf die Zahl folgt (z.B. „Die Windkraftanlage hat eine Leistung von 3 MW.“).

\item	\textbf{Passiv verwenden} \\
Es ist darauf zu achten, das Passiv an Stelle der „Ich-Form“ zu verwenden. Bsp.: „Es wurde untersucht...“ statt „Ich habe untersucht“.
\end{itemize}





\newpage
\textbf{Besonderheiten bei Forschungsprojekten und Studien- bzw. Projektarbeiten}
Bei der schriftlichen Ausarbeitung von Forschungsprojekten, Studien- und Projektarbeiten ergeben sich Umfang und inhaltliche Tiefe durch die in der jeweiligen Prüfungsordnung dafür festgelegten Leistungspunkte.
Handelt sich um ein mit 10 Leistungspunkten (oder 12 in Verbindung mit dem Besuch eines Seminars) gewichtetes Forschungsprojekt, sollte die schriftliche Ausarbeitung vom Niveau her knapp unter einer Bachelorarbeit einzuordnen sein. Die Abgabe erfolgt in diesem Fall in gebundener Form.
Für Studien- und Projektarbeiten, die mit 6 Leistungspunkten gewichtet sind, ist die schriftliche Ausarbeitung in Form einer wissenschaftlichen Veröffentlichung (Paper) anzufertigen. Sollte das Forschungsprojekt mit einem erheblichen praktischen Aufwand verbunden sein, kann auch in diesem Fall, bei vorheriger Absprache, die Abgabe in Form einer wissenschaftlichen Veröffentlichung erfolgen.




