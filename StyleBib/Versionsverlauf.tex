% Version 9: Dezmeber 2023
% Fixes:
% Es werden in der deutschen Version keine Spaces mehr zwischen Zahl und Dezimalkomma eingebaut.  
%%% Betroffene Dateien
% Packages.tex

% Version 8: November 2023
% Fixes:
% Behoben wurde das Problem, dass die Formelzeichentabelle nicht automatisch auf eine zweite Seite umbricht. Stefan Möws wurde als Ansprechpartner entfernt. 
%%% Betroffene Dateien
% main.tex
% formelzeichen.tex
% Packages.tex


% Version 7: April 2023
% Fixes:
% Ergänzung der stundentischen Informationen in der Abschlussarbeit um den Studiengang 
%%% Betroffene Dateien
% main.tex
% Titelseite.tex
% Einstellungen.tex



% Version 6: März 2023
% Fixes:
% Versionseintrag gelöscht in documentclass
%%% Betroffene Dateien
% main.tex

% Version 5: Juli 2022
% Fixes:
% Verwendung von anderen Tex-Versionen
%%% Betroffene Dateien
% Packages.tex



% Version 4: Juni 2022
% Fixes:
% Die Tabelle für Formelzeichen wird über mehr als nur eine Seite angezeigt (\usepackage{ltablex} eingefügt
%%% Betroffene Dateien
% Packages.tex



% Version 3: Juni 2022
%% Fixes: 
% Entfernen des doppelten Literaturverzeichnisses
% Einfügen von Examiner und Supervisor für englischsprachige Arbeiten
% Hinzufügen der Einheit Ohm und bar
% Entfernen von V aus Beschriftung von Abbildung 2.2
% Einfügen von \SI für Werte in Tabelle 2.1, damit entsprechend Trenner der Dezimalzahl für Englisch und Deutsch angepasst werden.
% Anpassung des SI packages für EN = Englisch und DE= Deutsch sowie einiger Formatierungen der SI Blöcke. 
% Detaillierung des Formelzeichenverzeichnisses (alphabetisch, lateinisch vor griechisch vor chemisch)
% In bib-Style die "Dashification" bei Wiederholung des Autors abgestellt.
%%% Betroffene Dateien
% Titelseite.tex
% Packages.tex
% main.tex
% Generatorleistung.tex
% Kapitel 2.tex
% Formelzeichen.tex
% IEEEtrans_DE.tex
% IEEEtrans_EN.tex


% Version 2: Juni 2022
%% Fixes: 
% Einstellung in Bib-Style, dass noch vier Autoren et al. folgt und drei Autoren vor dem et al. genannt werden. 
% Einfügen des siunitx-packages um besser auf die Anforderungen der Si schreibung einzugehen
% Entfernen des packages ziffer, da es zu Problemen mit siunitx kommt
% Hinzufügen von Dezimalmarker für SI-Einheiten in deutscher Sprache
% Einfügen des Packages nccmath, um chemische Formeln zentrieren zu können.
%%% Betroffene Dateien
% IEEEtrans_DE.tex
% IEEEtrans_EN.tex
% Packages.tex

% Version 1: Mai 2022 - 1. Entwurf für neues Template